\vspace*{\fill}
\pagenumbering{gobble}
\thispagestyle{plain}
\begingroup

    \begin{center}
        {{\large{\bf Abstract}}}
    \end{center}

    %\vspace{10mm}

    \begin{center}
        Advanced, optimized and well-performed data quality monitoring tasks in modern physics experiments are becoming
        essential, as the complexity of collected data increase with the development of more sophisticated detectors. In
        this thesis work, we have exploited neural networks to test the beginnings of an automated data quality
        monitoring procedure on drift tubes data. The algorithm uses statistical hypothesis testing to compare the
        monitored data sample with a reference dataset and returns information on the level of agreement between the
        two. The performance tests we have carried out show that our implementation is correctly working: the neural
        network detects anomalies in data when they are present and satisfactorily recognizes when data is of good
        quality.
        
    \end{center}

    \vspace{40mm}

    \begin{center}
        {{\large{\bf Sommario}}}
    \end{center}

    %\vspace{10mm}

    \begin{center}
        Lo sviluppo di rivelatori sempre più sofisticati per i moderni esperimenti di fisica, ed il conseguente aumento
        della complessità dei dati raccolti, richiede tecniche avanzate e ottimizzate di controllo qualità dei dati, le
        quali ricoprono perciò un ruolo cruciale. In questo lavoro di tesi, per testare un'innovativa procedura
        automatica di monitoraggio dati, sono state sfruttate reti neurali applicandole a dati provenienti da camere a
        deriva. L'algoritmo implementato utilizza test di ipotesi per comparare i dati da monitorare rispetto ad un
        campione di riferimento, restituendo informazioni riguardo il livello di accordo tra i due. Sono stati
        effettuati test per verificare le prestazioni dell'algoritmo: i risultati mostrano che, quando nei dati sono
        presenti anomalie qualitative, la rete è in grado di rilevarle e quantificarle. In caso contrario, il modello
        riconosce in modo soddisfacente l'accordo con il campione di riferimento.
    \end{center}

\endgroup

\vspace*{\fill}

