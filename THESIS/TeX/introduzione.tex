\chapter*{Introduction}
\addcontentsline{toc}{chapter}{Introduction} \markboth{INTRODUCTION}{}
\label{chap:Introduction}

The realm of this thesis work is data quality monitoring (DQM for short), which deals with being able to tell whether
data in a given dataset have been appropriately collected so that we can extrapolate relevant results from them, or acquired
poorly, and in the latter case, datasets may also be useless in their entirety. As experiments grow more extensive, and a larger
amount of data is being collected, the already crucial role of DQM turns out to be essential in modern physics
experiments. 

Data quality monitoring tasks have always been performed since the very first experiments: the scientist tests the
experimental setup on a well-known configuration, then visually compares the collected data with his expectations and,
if it looks like to him that the setup is working correctly, the scientist begins with his experiment. Of course,
through time, DQM procedures have evolved consistently to keep up with the ever-increasing complexity of the experiments
and the data being acquired. However, it is pretty common to encounter many human performed tasks, from visually
comparing data to running advanced statistical tests. As datasets get more extensive and more complex, it could become a
highly demanding and time-consuming job.  

Our main goal is thus to minimize the human efforts in data quality monitoring tasks, and we intend to exploit
machine learning techniques. Machine learning is the science of getting computers to perform operations without being
explicitly programmed. It is a branch of artificial intelligence that aims to imitate the way humans learn, gradually
improving its accuracy. It has become a most pervasive topic, finding applications in every branch of science and playing a substantial role in modern experimental physics. \\

In this thesis work, we have exploited a deep learning algorithm based on the maximum likelihood principle: it compares
the collected data with a given reference dataset and returns the level of agreement between the two in the form of a
test statistic. It has been tested on data quality monitoring tasks using datasets coming from the realm of particle
physics. Precisely, we have been working on a cosmic muon telescope, a drift tubes detector, located at the Legnaro INFN
National Laboratories and reproducing a small-scale replica of the CMS's muon chambers. The experimental setup and the
employed datasets are introduced and described in \autoref{chap:Experiment}. In \autoref{chap:Algorithm} we offer a
brief explanation of the algorithm, providing a brief yet crucial conceptual background and a general overview of its
functioning. Lastly, the DQM performance of the algorithm is reported in \autoref{chap:Results}, along with its
technical implementation and prospects. 




