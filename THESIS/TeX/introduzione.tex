\chapter{Introduction}
% \addcontentsline{toc}{chapter}{Introduction} \markboth{INTRODUCTION}{}
\label{chap:Introduction}

\section{Machine Learning applications in High Energy Physics}

Particle Physics, also known as High Energy Physics (HEP), is the branch of physics that studies the nature of the
particles that costitute matter and radiation. Its development can be located starting from the second half of the
$19^{\text{th}}$ century and, since then, theoretical physicists have been working on models that can accurately predict
and describe the outcome of experiments. The model that better describes the experimental results is the Standard Model
(SM), although it is common knowledge among the HEP community that SM is not complete. As a matter of fact, it does not
describe General Relativity in terms of quantum field theories, it does not give specifics about nautrino masses and
does not explain the existance of dark matter. These facts show that there are as yet undiscovered physical laws and
that our understanding of the world at its most fundamental level is lacking.\\

In order to interrogate experimental data in the search for New Physics (NP), scientists have come to realise that
employing a model-dependet approach (i.e. searching for specific NP models) has a critical disadvantage: a statistical
test which is designed to be sensitive to one specific hypothesis is typically insensitive to data departures of a
different nature from the one expected. This means that if new physics is present in the data, but not predicted by the
specific NP model that is being tested, it would not be discovered.\\

A modern approach that overcomes the difficulties underlined above is the so called model-indipendent approach. For the
sake of clarity, in the context of statistics, testing one hypothesis requires an alternative hypothesis to be compared
with. In physics, a model is a set of physical laws that make us able to predict the distribution of a certain feature
depending on free parameters. We thus define a certain approach 'model-independent' if the alternative distributions do
not follow from a physical model. It means that the distributions can adapt to the true underlying data distribution for
an appropriate choice of the free parameters. The main reason for which we demand a model-independent approach is the
advantage of sensitivity to a large variety of new physics scenarios.\\

In the realm of HEP, Machine Learning (ML) techniques has been used since early $2000$ (traditionally known as
Multivariate Analysis) in many crucial tasks, such as event classification, track reconstruction and particle
identification. 



\section{Data Quality Monitoring}

